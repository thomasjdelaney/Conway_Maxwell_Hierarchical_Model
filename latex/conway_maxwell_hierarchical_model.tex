\documentclass[a4paper,12pt]{article}
\usepackage[utf8x]{inputenc}
\usepackage{amssymb}
\usepackage{amsfonts}
\usepackage{mathrsfs}
\usepackage{amsmath}
\usepackage{amsthm}
\usepackage[margin=3cm]{geometry}
\usepackage{times}
\usepackage{graphicx}
\usepackage{enumitem}
\usepackage{fancyhdr}
\usepackage{hyperref}
\usepackage{setspace}
\usepackage{subcaption}
\usepackage{mathtools}

\usepackage{lineno}
\linenumbers
\renewcommand{\baselinestretch}{1.5}
\usepackage{authblk}

\newtheorem{theorem}{Theorem}
\newtheorem{proposition}{Proposition}[section]
\newtheorem{lemma}{Lemma}[section]
\newtheorem{corollary}{Corollary}[section]
\theoremstyle{definition}
\newtheorem{definition}{Definition}[section]

\newcommand{\boldnabla}{\mbox{\boldmath$\nabla$}} % to be used in mathmode
\newcommand{\cbar}{\overline{\mathbb{C}}}% to be used in mathmode
\newcommand{\diff}[2]{\frac{d #1}{d #2}}% to be used in mathmode
\newcommand{\difff}[2]{\frac{d^2 #1}{d #2^2}}% to be used in mathmode
\newcommand{\pdiff}[2]{\frac{\partial #1}{\partial #2}} % to be used in mathmode
\newcommand{\pdifff}[2]{\frac{\partial^2 #1}{\partial #2^2}}% to be used in mathmode
\newcommand{\upperth}{$^{\mbox{\footnotesize{th}}}$}%to be used in text mode
\newcommand{\vect}[1]{\mathbf{#1}}% to be used in mathmode
\newcommand{\curl}[1]{\boldnabla \times \vect{#1}} % to be used in mathmode
\newcommand{\divr}[1]{\boldnabla \cdot \vect{#1}} %to be used in mathmode
\newcommand{\modu}[1]{\left| #1 \right|} %to be used in mathmode
\newcommand{\brak}[1]{\left( #1 \right)} % to be used in mathmode
\newcommand{\comm}[2]{\left[ #1 , #2 \right]} %to be used in mathmode
\newcommand{\dop}{\vect{d}} %to be used in mathmode
\newcommand{\cov}{\text{cov}} %to be used in mathmode
\newcommand{\var}{\text{var}} %to be used in mathmode
\newcommand{\mb}{\mathbf} %to be used in mathmode
\newcommand{\bs}{\boldsymbol} %to be used in mathmode
% Title Page
\title{A simple two parameter distribution for modelling neuronal activity and capturing neuronal association}
\date{}

\author[1]{Thomas Delaney}
\author[1]{Cian O'Donnell}
\affil[1]{School of Computer Science, Electrical and Electronic Engineering, and Engineering Mathematics, University of Bristol, Bristol, United Kingdom.}
\renewcommand\Affilfont{\itshape\small}

\begin{document}

\maketitle

%\tableofcontents

\abstract{Recent developments in electrophysiological technology have lead to an increase in the size of electrophysiological datasets. Consequently, there is a requirement for new analysis techniques that can make use of these new datasets, while remaining easy to use in practice. In this work, we fit the Conway-Maxwell-binomial distribution to spiking data read from a mouse exposed to visual stimuli.}

\section{Introduction}
  Motivate by pointing out how much computational power it can require to calculate $n$th order correlations.

  Point out that we don't necessarily need to measure correlations anyway.

\section{Results}

\section{Discussion}

\section{Data}

Details from data cortex lab here.

\section{Methods}
Details about all kinds of things here.
    \subsection{Binning data}
    We binned the spiking data into very small bins.

\section{Discussion}
Point out that the Conway-Maxwell-binomial distribution could be used to measure activity and association without having to sort the voltage traces into spikes. That does defeat the purpose slightly, however. 

\newpage
\bibliography{conway_maxwell_hierarchical_model.bbl}

\end{document}
