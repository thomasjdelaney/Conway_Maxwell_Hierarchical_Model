\documentclass[a4paper,12pt]{article}
\usepackage[utf8x]{inputenc}
\usepackage{amssymb}
\usepackage{amsfonts}
\usepackage{mathrsfs}
\usepackage{amsmath}
\usepackage{amsthm}
\usepackage[margin=3cm]{geometry}
\usepackage{times}
\usepackage{graphicx}
\usepackage{enumitem}
\usepackage{fancyhdr}
\usepackage{hyperref}
\usepackage{setspace}
\usepackage{subcaption}
\usepackage{mathtools}

\usepackage{lineno}
\linenumbers
\renewcommand{\baselinestretch}{1.5}
\usepackage{authblk}

\newtheorem{theorem}{Theorem}
\newtheorem{proposition}{Proposition}[section]
\newtheorem{lemma}{Lemma}[section]
\newtheorem{corollary}{Corollary}[section]
\theoremstyle{definition}
\newtheorem{definition}{Definition}[section]

\newcommand{\boldnabla}{\mbox{\boldmath$\nabla$}} % to be used in mathmode
\newcommand{\cbar}{\overline{\mathbb{C}}}% to be used in mathmode
\newcommand{\diff}[2]{\frac{d #1}{d #2}}% to be used in mathmode
\newcommand{\difff}[2]{\frac{d^2 #1}{d #2^2}}% to be used in mathmode
\newcommand{\pdiff}[2]{\frac{\partial #1}{\partial #2}} % to be used in mathmode
\newcommand{\pdifff}[2]{\frac{\partial^2 #1}{\partial #2^2}}% to be used in mathmode
\newcommand{\upperth}{$^{\mbox{\footnotesize{th}}}$}%to be used in text mode
\newcommand{\vect}[1]{\mathbf{#1}}% to be used in mathmode
\newcommand{\curl}[1]{\boldnabla \times \vect{#1}} % to be used in mathmode
\newcommand{\divr}[1]{\boldnabla \cdot \vect{#1}} %to be used in mathmode
\newcommand{\modu}[1]{\left| #1 \right|} %to be used in mathmode
\newcommand{\brak}[1]{\left( #1 \right)} % to be used in mathmode
\newcommand{\comm}[2]{\left[ #1 , #2 \right]} %to be used in mathmode
\newcommand{\dop}{\vect{d}} %to be used in mathmode
\newcommand{\cov}{\text{cov}} %to be used in mathmode
\newcommand{\var}{\text{var}} %to be used in mathmode
\newcommand{\mb}{\mathbf} %to be used in mathmode
\newcommand{\bs}{\boldsymbol} %to be used in mathmode
% Title Page
\title{A simple two parameter distribution for modelling neuronal activity and capturing neuronal association}
\date{}

\author[1]{Thomas Delaney}
\author[1]{Cian O'Donnell}
\affil[1]{School of Computer Science, Electrical and Electronic Engineering, and Engineering Mathematics, University of Bristol, Bristol, United Kingdom.}
\renewcommand\Affilfont{\itshape\small}

\begin{document}

\maketitle

%\tableofcontents

\abstract{Recent developments in electrophysiological technology have lead to an increase in the size of electrophysiological datasets. Consequently, there is a requirement for new analysis techniques that can make use of these new datasets, while remaining easy to use in practice. In this work, we fit the Conway-Maxwell-binomial distribution to spiking data read from a mouse exposed to visual stimuli.}

\section{Introduction}
  Motivate by pointing out how much computational power it can require to calculate $n$th order correlations.

  Point out that we don't necessarily need to measure correlations anyway.

\section{Results}

\section{Discussion}

\section{Data}
We used data collected by Nick Steinmetz and his lab `CortexLab at UCL' \cite{steinmetz}. The data can be found online \footnote{\url{http://data.cortexlab.net/dualPhase3/}} and are free to use for research purposes.

Two `Phase3' Neuropixels \cite{jun} electrode arrays were inserted into the brain of an awake, head-fixed mouse for about an hour and a half. These electrode arrays recorded $384$ channels of neural data each at $30$kHz and  less than $7$µV RMS noise levels. The sites are densely spaced in a `continuous tetrode'-like arrangement, and a whole array records from a $3.8$mm span of the brain. One array recorded from visual cortex, hippocampus,  and thalamus, the other array recorded from motor cortex and striatum. The data were spike-sorted automatically by Kilosort and manually by N. Steinmetz using Phy. In total 831 well-isolated individual neurons were identified.

  \subsection{Experimental protocol}
  The mouse was shown a visual stimulus on three monitors placed around the mouse at right angles to each other, covering about $\pm 135$ degrees azimuth and $\pm 35$ degrees elevation.

  The stimulus consisted of sine-wave modulated full-field drifting gratings of 16 drift directions ($0^{\circ}, 22.5^{\circ}, \dots, 337.5^{\circ}$) with $2$Hz temporal frequency and $0.08$ cycles/degree spatial frequency displayed for $2$ seconds plus a blank condition. Each of these $17$ conditions were presented $10$ times in a random order across $170$ different trials. There were therefore $160$ trials with a visual stimulus present, and $10$ trials without a visual stimulus.

\section{Methods}
Details about all kinds of things here.
    \subsection{Binning data}
    We converted the spike times for each cell into spike counts by putting the spike times into time bins of a given `width' (in seconds). We used time bins of $1$ms, $5$ms, and $10$ms. We used different time bin widths to assess the impact of choosing a bin width. 


    \subsection{Number of \textit{active} neurons}
    To count the number of active neurons in each neuronal ensemble, we split the time interval for each trial into bins of a given width. We counted the number of spikes fired by each cell in each bin. If a cell fired \textit{at least} one spike in a given bin, we regarded that cell as active in that bin. We recorded the number of active cells in every bin, and we recorded each cell's individual spike counts.
    
    It should be noted that when we used a bin width of $1$ms, the maximum number of spikes in any bin was $1$. For the wider time bins, some bins had spike counts greater than $1$. Consequently when using a bin width of $1$ms, the number of active neurons and the total spike count of a given bin were identical. But for wider bin widths, the total spike count was greater than the number of active neurons. 

    So for the $1$ms bin width, the activity of a neuron and the number of spikes fired by that neuron in any bin can be modelled as a Bernoulli variable. But for wider time bins, only the activity can be modelled in this way.

    \subsection{Moving windows for measurements}

    When taking measurements (e.g. moving average over the number of active neurons) or fitting distributions (eg. the beta binomial distribution) we slid a window containing a certain number of bins across the data, and made our measurements at each window position. For example, when analysing $1$ms bin data, we used a window containing $100$ bins, and we slid the window across the time interval for each trial moving $10$ bins at a time. So that for $2560$ms of data, we made $246$ measurements.

    For the $5$ms bin width data, we used windows containing $40$ bins, and slid the window $2$ bins at a time when taking measurements.

    For the $10$ms bin width data, we used windows containing $40$ bins, and slid the window $1$ bin at a time when taking measurements.
    
    By continuing to use windows containing $40$ bins, we retained statistical power but sacrificed the number of measurements taken. 

    \subsection{Fano factor}\label{sec:faco_factor}
    The \textit{Fano factor} of a random variable is defined as the ratio of the variable's variance to its mean. 
    \begin{align}\label{eq:fano_factor}
      F = \frac{\sigma^2}{\mu}
    \end{align}
    We measured the Fano factor of the spike count of a given cell by measuring the mean and variance of the spike count across trials, and taking the ratio of those two quantities. When calculated in this way the Fano factor can be used as a measure of neural variability.  This is similar to the calculation used in \cite{churchland}.

    \subsection{Probability Distributions suitable for modelling ensemble activty}

      \subsubsection{Binomial distribution}
      The binomial distribution is a discrete probability distribution that can be thought of as the sum of independent Bernoulli random variables, each with the same probability of success. The binomial distribution has two parameters, $n$, the number of Bernoulli variables, and $p$, the probability of success for each of these variables. A random variable with the binomial distribution can take values  from $\lbrace 0, \dots, n \rbrace$. The probability mass function is
      \begin{align}\label{eq:binomial_pmf}
        P(k;n,p) = \binom{n}{k} p^k (1-p)^{n-k}
      \end{align}

      If we have $N$ samples from a binomial distribution and we know $n$ but want to estimate $p$, we can maximise the log likelihood function
      \begin{align}\label{eq:binomial_log_like}
        L(p) & = \log P(\lbrace k_1, \dots, k_N \rbrace; n,p) \\ 
            & = \sum_{i=1}^N \log \binom{n}{k_i} + k_i \log p + (n-k_i)\log(1-p) 
      \end{align}

      If we do not know $n$ there is no closed form way way of maximising this equation. Therefore the binomial distribution is generally only used in cases where we do know $n$. Consequently, the distribution is practically a one parameter distribution.

      \subsubsection{Beta-binomial distribution}

      \subsubsection{Conway-Maxwell-binomial distribution}

    \subsection{Goodness-of-fit}

    \subsection{Spike count correlations}

\section{Discussion}
Point out that the Conway-Maxwell-binomial distribution could be used to measure activity and association without having to sort the voltage traces into spikes. That does defeat the purpose slightly, however. 

\newpage
\bibliography{conway_maxwell_hierarchical_model.bbl}

\end{document}
